\begin{mdframed}
    \textbf{La extensión máxima para esta sección es de 2 páginas.}
\end{mdframed}

El área de Análisis y Diseño de algoritmos en Ciencias de la Computación nos ha ayudado a mejorar nuestras vidas a través de la resolución y rápidos cálculos de varios problemas propuestos del trabajo y de la vida cotidiana, además de aplicaciones de las mismas.

Sin embargo, hay que dar un paso hacia atrás y preguntarnos, ¿cómo se plasman en la práctica estas soluciones, estos algoritmos? ¿Cómo resultan ser de rápidos y eficientes en el mundo real?

En particular, vamos a echar un vistazo a los algoritmos de fuerza bruta y de programación dinámica (dynamic programming) para encontrar las diferencias entre dos strings cualquiera; más específicamente, usaremos implementaciones tanto de fuerza bruta como de DP del algoritmo de LCS (Longest Common Substring) para determinar las diferencias entre los strings. Esto nos va a permitir ver los beneficios de ambos algoritmos tanto en memoria y en tiempo, no sólo en la teoría, si no que en la práctica también, según varios tamaños posibles de strings.