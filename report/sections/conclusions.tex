\begin{mdframed}
    \textbf{La extensión máxima para esta sección es de 1 página.}
\end{mdframed}

Como conclusión, si bien el algoritmo de fuerza bruta sirve para strings chicos, el tiempo de ejecución se dispara totalmente al empezar a trabajar con strings medianos y largos. Se demuestra la superioridad del algoritmo de programación dinámica en este ámbito, sin embargo queda en muestra también su gran debilidad, que es su uso de memoria y la incomputabilidad resultante del algoritmo para strings larguísimos (un caso hipotético de esto siendo la comparación entre dos guiones de película para detectar plagio), por lo que si bien tarda mucho más en llegar a un bloque práctico, este existe.

Al fin y al cabo, no hay algoritmo de oro para un problema; se debe seleccionar el algoritmo a usar en base a las condiciones del input, qué tan rápido se desea el resultado, nuestras limitaciones de memoria, etc.

