\begin{mdframed}
    \textbf{La extensión máxima para esta sección es de 5 páginas.}
\end{mdframed}

Para el problema de encontrar diferencias en strings, se ha determinado usar una versión modificada del algoritmo de Longest Common Substring, o LCS, seguido de un recorrido lineal de ambos strings en base al resultado arrojado por el algoritmo. En este caso, la modifiación del algoritmo consiste en arrojar los substrings comunes directamente en vez de los tamaños de estos.

Existen dos implementaciones de este algoritmo; fuerza bruta, y programación dinámica. Notar que su pseudocódigo será presentado en su versión no modificada. Ambos pseudocódigos están sacados de la siguiente fuente: \href{https://www.cubawiki.com.ar/images/1/1a/Lcs_tagliavini.pdf}{Algoritmos y Estructuras de Datos III - Práctica: Programación Dinámica, Guido Tagliavini Ponce, 03/09/2014}.