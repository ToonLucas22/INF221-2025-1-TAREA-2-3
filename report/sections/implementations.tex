\begin{mdframed}
    \textbf{La extensión máxima para esta sección es de 1 página.}
\end{mdframed}
\begin{mdframed}
    Cada algoritmo debe implementarse en el lenguaje de programación C++, siguiendo los formatos de entrada y salida.
\end{mdframed}
Tenemos ambos algoritmos almacenados en carpetas separadas dentro de la carpeta code, junto con sus input, sus generadores de input, sus plot generators, sus output, etc.

Adentrándose a una de las dos carpetas nos encontramos con las carpetas algorithm, data, y scripts, ademas del programa principal y su makefile. En algorithm se ubica el algoritmo en sí, en data se encuentra todo lo relevante a lo que son los input, los output, las medidas de tiempo, y los plots gráficos de medición, y finalmente en scripts se ubican los scripts de Python que generan los input y los plots para el algoritmo.

Las implementaciones de ambos algoritmos son similares al pseudocódigo presentado, pero modificado para retornar substrings en vez de sólo números, esto con el fin de tener a mano el substring común más largo en vez de solamente el tamaño de éste, y así poder hacer un recorrido lineal de ambos strings junto con el substring obtenido, con el objetivo de obtener directamente las diferencias de los strings.

El código del proyecto se ubica en el enlace siguiente:

\begin{mdframed}
    \begin{center}
        {\Large \url{https://github.com/ToonLucas22/INF221-2025-1-TAREA-2-3/tree/main}}
    \end{center}
\end{mdframed}