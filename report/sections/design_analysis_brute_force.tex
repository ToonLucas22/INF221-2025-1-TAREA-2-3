\epigraph{\textit{``Indeed, brute force is a perfectly good technique in many cases; the real question is, can we use brute force in such a way that we avoid the worst-case behavior?''}}{--- \citeauthor{taocv3}, \citeyear{taocv3} \cite{taocv3}}

\begin{algorithm}[H]
    \SetKwProg{myproc}{Procedure}{}{}
    \SetKwFunction{AlgorithmName}{LCS-Naive}  % Cambia 'AlgorithmName' por el nombre del enfoque elegido
    
    \DontPrintSemicolon
    \footnotesize

    % Definición del algoritmo principal
    \myproc{\AlgorithmName{S1(x1, ..., xi), S2(y1, ..., yj}}{
    \uIf{i = 0 or j = 0}{
        \Return 0\;  % Return explícito si S1 está vacía
    }
    \uElseIf{xi = yj}{
        \Return 1 + \AlgorithmName{S1(i-1), S2(j-1)}  % Llamada recursiva
    }
    \Else{
        % Ejemplo de llamado a una función auxiliar
        \Return máx(\AlgorithmName{S1, S2(j-1)}, \AlgorithmName{S1(i-1), S2})
    }
    }

    \caption{Implementación de LCS a la fuerza bruta, basado en \href{https://www.cubawiki.com.ar/images/1/1a/Lcs_tagliavini.pdf}{Algoritmos y Estructuras de Datos III - Práctica: Programación Dinámica, Guido Tagliavini Ponce, 03/09/2014}}
    \label{alg:mi_algoritmo_1}
\end{algorithm}
